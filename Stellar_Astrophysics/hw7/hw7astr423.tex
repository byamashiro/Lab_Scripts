%%
%% Beginning of file 'sample.tex'
%%
%% Modified 2015 December
%%
%% This is a sample manuscript marked up using the
%% AASTeX v6.x LaTeX 2e macros.

%% AASTeX is now based on Alexey Vikhlinin's emulateapj.cls 
%% (Copyright 2000-2015).  See the classfile for details.
%%
%% AASTeX requires revtex4-1.cls (http://publish.aps.org/revtex4/) and
%% other external packages (latexsym, graphicx, amssymb, longtable, and epsf).
%% All of these external packages should already be present in the modern TeX 
%% distributions.  If not they can also be obtained at www.ctan.org.

%% The first piece of markup in an AASTeX v6.x document is the \documentclass
%% command. LaTeX will ignore any data that comes before this command. The 
%% documentclass can take an optional argument to modify the output style.
%% The command below calls the preprint style  which will produce a tightly 
%% typeset, one-column, single-spaced document.  It is the default and thus
%% does not need to be explicitly stated.
%%

%% using aastex version 6
\documentclass[onecolumn]{aastex6}
\usepackage{subfigure}
\usepackage{amsmath}
\usepackage{listings}
\usepackage{color}
 
\definecolor{codegreen}{rgb}{0,0.6,0}
\definecolor{codegray}{rgb}{0.5,0.5,0.5}
\definecolor{codepurple}{rgb}{0.58,0,0.82}
\definecolor{backcolour}{rgb}{0.95,0.95,0.92}
 
\lstdefinestyle{mystyle}{
    backgroundcolor=\color{backcolour},   
    commentstyle=\color{codegreen},
    keywordstyle=\color{magenta},
    numberstyle=\tiny\color{codegray},
    stringstyle=\color{codepurple},
    basicstyle=\footnotesize,
    breakatwhitespace=false,         
    breaklines=true,                 
    captionpos=b,                    
    keepspaces=true,                 
    numbers=left,                    
    numbersep=5pt,                  
    showspaces=false,                
    showstringspaces=false,
    showtabs=false,                  
    tabsize=2
}
 
\lstset{style=mystyle}


%% The other main article choice is a tightly typeset, two-column article
%% that more closely resembles the final typeset pdf article.
%%
%% \documentclass[twocolumn]{aastex6}
%% 
%% There are other optional arguments one can envoke to allow other 
%% actions. 
%%
% These are the available options:
%   manuscript	: onecolumn, doublespace, 12pt fonts
%   preprint	: onecolumn, single space, 10pt fonts
%   preprint2	: twocolumn, single space, 10pt fonts
%   twocolumn	: a two column article. Probably not needed, but here just in case.
%   onecolumn	: a one column article; default option.
%   twocolappendix: make 2 column appendix
%   onecolappendix: make 1 column appendix is the default. 
%   astrosymb	: Loads Astrosymb font and define \astrocommands. 
%   tighten	: Makes baselineskip slightly smaller
%   times	: uses times font instead of the default
%   linenumbers	: turn on lineno package.
%   trackchanges : required to see the revision mark up and print output
%   numberedappendix: Labels appendix sections A, B, ... This is the default.
%   appendixfloats: Needed. Resets figure and table counters to zero

%% these can be used in any combination, e.g.
%%
%% \documentclass[twocolumn,twocolappendix,linenumbers,trackchanges]{aastex6}

%% If you want to create your own macros, you can do so
%% using \newcommand. Your macros should appear before
%% the \begin{document} command.
%%
\newcommand{\vdag}{(v)^\dagger}
\newcommand\aastex{AAS\TeX}
\newcommand\latex{La\TeX}

%% AASTeX 6.0 supports the ability to suppress the names and affiliations
%% of some authors and displaying them under a "collaboration" banner to
%% minimize the amount of author information that to be printed.  This 
%% should be reserved for articles with an extreme number of authors.
%%
%% Mark up commands to limit the number of authors on the front page.
\AuthorCallLimit=2
%% Will only show Schwarz & Muench since Schwarz and Muench
%% are in the same \author call. 
\fullcollaborationName{The Friends of AASTeX Collaboration}
%% will print the collaboration text after the shortened author list.
%% These commands have to COME BEFORE the \author calls.
%%
%% Note that all of these author will be shown in the published article.
%% This feature is meant to be used prior to acceptance to make the
%% front end of a long author article more manageable.
%% Use \allauthors at the manuscript end to show the full author list.

%% The following command can be used to set the latex table counters.  It
%% is needed in this document because it uses a mix of latex tabular and
%% AASTeX deluxetables.  In general it should not be needed.
%\setcounter{table}{1}

%%%%%%%%%%%%%%%%%%%%%%%%%%%%%%%%%%%%%%%%%%%%%%%%%%%%%%%%%%%%%%%%%%%%%%%%%%%%%%%%
%%
%% The following commented section outlines numerous optional output that
%% can be displayed in the front matter or as running meta-data.
%%
%% You can insert a short comment on the title page using the command below.
%% \slugcomment{Not to appear in Nonlearned J., 45.}
%%
%% If you wish, you may supply running head information, although
%% this information may be modified by the editorial offices.
%%\shorttitle{\aastex sample article}
%%\shortauthors{Schwarz et al.}
%%
%% You can add a light gray and diagonal water-mark to the first page 
%% with this command:
%% \watermark{text}
%% where "text", e.g. DRAFT, is the text to appear.  If the text is 
%% long you can control the water-mark size with:
%% \setwatermarkfontsize{dimension}
%% where dimension is any recognized LaTeX dimension, e.g. pt, in, etc.
%%
%%%%%%%%%%%%%%%%%%%%%%%%%%%%%%%%%%%%%%%%%%%%%%%%%%%%%%%%%%%%%%%%%%%%%%%%%%%%%%%%

%% This is the end of the preamble.  Indicate the beginning of the
%% paper itself with \begin{document}.

\begin{document}

%% LaTeX will automatically break titles if they run longer than
%% one line. However, you may use \\ to force a line break if
%% you desire.

\title{HW 07: Equations of State and the Temperature-Density Plane}

%% Use \author, \affil, plus the \and command to format author and affiliation 
%% information.  If done correctly the peer review system will be able to
%% automatically put the author and affiliation information from the manuscript
%% and save the corresponding author the trouble of entering it by hand.
%%
%% The \affil should be used to document primary affiliations and the
%% \altaffil should be used for secondary affiliations, titles, or email.

%% Authors with the same affiliation can be grouped in a single
%% \author and \affil call.
\author{Bryan Yamashiro\altaffilmark{1}}
\affil{University of Hawaii at Manoa \\
2500 Campus Road \\
Honolulu, HI 96822}


%% Use the \and command so offset the last author.

%% Notice that each of these authors has alternate affiliations, which
%% are identified by the \altaffilmark after each name.  Specify alternate
%% affiliation information with \altaffiltext, with one command per each
%% affiliation.

%\altaffiltext{1}{A cool dude}
%\altaffiltext{2}{Another cool dude}


%% From the front matter, we move on to the body of the paper.
%% Sections are demarcated by \section and \subsection, respectively.
%% Observe the use of the LaTeX \label
%% command after the \subsection to give a symbolic KEY to the
%% subsection for cross-referencing in a \ref command.
%% You can use LaTeX's \ref and \label commands to keep track of
%% cross-references to sections, equations, tables, and figures.
%% That way, if you change the order of any elements, LaTeX will
%% automatically renumber them.

%% We recommend that authors also use the natbib \citep
%% and \citet commands to identify citations.  The citations are
%% tied to the reference list via symbolic KEYs. The KEY corresponds
%% to the KEY in the \bibitem in the reference list below. 

\section{Different Formulas for Pressure}

Radiation Pressure
\begin{equation}
P_{rad} = \frac{4\sigma T^4}{3c}
\label{1}
\end{equation}

Ideal Gas Pressure
\begin{equation}
P_{ideal} = \frac{\rho k T}{\mu m_H}
\label{2}
\end{equation}

Non-Relativistic Electron Degeneracy Pressure
\begin{equation}
P = 10^{13}\left(\frac{\rho}{\mu_e}\right)^{5/3}
\label{3}
\end{equation}

Extremely Relativistic Electron Degeneracy Pressure
\begin{equation}
P = 1.245\times10^{15}\left(\frac{\rho}{\mu_e}\right)^{4/3}
\label{4}
\end{equation}

\section{Areas of Preponderance in the log(T) - log($\rho$) Plane}
The four equations\,1-4 for the different formulas for pressure were equated against each other for this section. Figure\,\ref{gasgraph} characterizes the four domain regions in in the T - log $\rho$ plane, where radiation pressure\,(1), ideal gas\,(2), degenerate gas\,(3), and relativistic gas\,(4) dominate. The four regions in figure\,\ref{gasgraph} includes the boundaries radiation-ideal, ideal-degenerate\,(non-relativistic), ideal-degenerate\,(relativistic), and degenerate\,(non-relativistic)-degenerate\,(relativistic).
\\
\indent Figure\,\ref{gasgraph} also demonstrates the core properties with a temperature at 15,000,000\,K and density of 150\,g\,cm$^{-3}$, which are indicated by the orange dashed lines and, more specifically, the blue circle. The plot shows that the core equation of state is most adequate in the \boxed{ideal-gas} pressure region.

\subsection{Equation of Boundaries}
Ideal Gas and Radiation Pressure Boundary
\begin{equation}
T_{ideal-rad} = \left(\frac{3c\rho k_B}{4 \sigma \mu m_H}\right)^{1/3}
\label{5}
\end{equation}

Ideal Gas and Non-Relativistic Electron Degeneracy Pressure Boundary
\begin{equation}
T_{ideal-nonrel} = \frac{\mu m_H 10^{13} \left(\frac{\rho}{\mu_e}\right)^{5/3}}{\rho k_B}
\label{6}
\end{equation}

Ideal Gas and Extremely Relativistic Electron Degeneracy Pressure Boundary
\begin{equation}
T_{ideal-rel} = \frac{\mu m_H 1.245\times10^{15} \left(\frac{\rho}{\mu_e}\right)^{4/3}}{\rho k_B}
\label{7}
\end{equation}

Non-Relativistic and Extremely Relativistic Degeneracy Pressure Boundary
\begin{equation}
\frac{\rho^{5/3}}{\rho^{4/3}} = \frac{1.245\times10^{15} \mu_e^{5/3}}{10^{13} \mu_e^{4/3}}
\label{8}
\end{equation}




\begin{figure*}[ht]
  \centering
  \includegraphics[scale=0.5]{plot4.png}%\quad
  \caption{Domains of the validity of the ideal-gas approximation, radiation pressure, degenerate gas, and relativistic degenerate gas. The colors represent the boundaries radiation-ideal\,[blue], ideal-degenerate\,(non-relativistic)\,[red],ideal-degenerate\,(relativistic)\,[green], degenerate\,(non-relativistic)-degenerate\,(relativistic)\,[purple]. The blue circle and orange dashed lines represents the temperature at 15,000,000\,K and a density of 150\,g\,cm$^{-3}$. The regimes are labeled are where each pressure scheme dominates, including radiation\,(1), ideal gas\,(2), non-relativistic degenerate gas\,(3), and extremely-relativistic degenerate gas\,(4).}
  \label{gasgraph}
\end{figure*}


\clearpage
\section{Appendix}

\begin{lstlisting}[language=Python, caption=Python source code.]
from astropy import units as u
from astropy.units import imperial as imp
#import astropy.units as u
import numpy as np
from astropy import constants as const
import matplotlib.pyplot as plt
import scipy
from scipy import special

mu_e = 1.5
mu = 0.85
k_1 = 1.00*10.0**7.0
k_2 = 1.24*10.0**11.0

rho = np.arange(10**(-9),10**9,1000.)
rho_non = np.arange(10**(-9),2.89467*10**6,1000.)
rho_rel = np.arange(2.89467*10**6,10**9,1000.)
#temp = np.arange(3,12,0.01)

inner_rho = (rho / mu_e)
inner_rho_non = (rho_non / mu_e)
inner_rho_rel = (rho_rel / mu_e)


T_1 = (((3.0 * const.c * const.k_B * rho)/(4.0 * const.sigma_sb * mu * const.u))**(1./3.)).cgs

#using book approximation (proportionality so ignore this case)
#T_2 = ((mu/(rho * const.k_B))*((rho/mu_e)**(5.0/3.0))).cgs
#T_3 = ((mu/(rho * const.k_B))*((rho/mu_e)**(4.0/3.0))).cgs

#online constants used to derive
#T_2 = k_1 * (rho**(5./3.))
#T_3 = k_2 * (rho**(4./3.))

#online constants used to derive
#T_2 = k_1 * (rho**(5./3.))
#T_3 = k_2 * (rho**(4./3.))

#slide constants
#T_2 = (10.0**13.0)*((rho/mu_e)**(5.0/3.0))
#T_3 = (1.245*(10.0**15.0))*((rho/mu_e)**(4.0/3.0))

T_2 = ((mu * const.u * (10.0**13.0) * (inner_rho_non ** (5.0/3.0)) ) / (rho_non * const.k_B)).cgs
T_3 = ((mu * const.u * (1.245*(10.0**15.0)) * (inner_rho_rel ** (4.0/3.0)) ) / (rho_rel * const.k_B)).cgs


#idx = np.argwhere(np.diff(np.sign(T_3 - T_2)) != 0).reshape(-1) + 0



plt.plot(rho, T_1, color='blue', linewidth = 3)
plt.plot(rho_non, T_2,  color='red', linewidth = 3)
plt.plot(rho_rel, T_3, color='green', linewidth = 3)

#idx = np.argwhere(np.isclose(T_2, T_3, atol=0.1)).reshape(-1)
#plt.plot(rho[idx], T_3[idx], 'ro')
#ax1.axvline(goes_proton_time[max_index], color='black', linewidth=1)
plt.vlines(x=2.89467*10.0**6.0, ymin = 0, ymax = 1*10.0**9.0, color='purple', linewidth = 3)
plt.axvline(x=150, ymin=0, ymax=10., hold=None,linestyle = '-.', color='orange')
plt.axhline(y=15000000.0, hold=None,linestyle = '-.', color='orange')
plt.plot(150, 15000000.0, 'ob')


#plt.plot(lambda_queue, d_gaussian, color='red')
plt.xlabel(r'log($ \rho $)', fontname="Arial", fontsize = 14)
plt.ylabel('log(T)', fontname="Arial", fontsize = 14)
plt.minorticks_on()
plt.ylim([10**3,10**12])
plt.xlim([10**(-9),10**(9)])
plt.grid(True)
plt.yscale('log')
plt.xscale('log')

plt.savefig('plot.pdf', format='pdf', dpi=900)

plt.show()
\end{lstlisting}



%% Appendix material should be preceded with a single \appendix command.
%% There should be a \section command for each appendix. Mark appendix
%% subsections with the same markup you use in the main body of the paper.

%% Each Appendix (indicated with \section) will be lettered A, B, C, etc.
%% The equation counter will reset when it encounters the \appendix
%% command and will number appendix equations (A1), (A2), etc.


%% The reference list follows the main body and any appendices.
%% Use LaTeX's thebibliography environment to mark up your reference list.
%% Note \begin{thebibliography} is followed by an empty set of
%% curly braces.  If you forget this, LaTeX will generate the error
%% "Perhaps a missing \item?".
%%
%% thebibliography produces citations in the text using \bibitem-\cite
%% cross-referencing. Each reference is preceded by a
%% \bibitem command that defines in curly braces the KEY that corresponds
%% to the KEY in the \cite commands (see the first section above).
%% Make sure that you provide a unique KEY for every \bibitem or else the
%% paper will not LaTeX. The square brackets should contain
%% the citation text that LaTeX will insert in
%% place of the \cite commands.

%% We have used macros to produce journal name abbreviations.
%% \aastex provides a number of these for the more frequently-cited journals.
%% See the Author Guide for a list of them.

%% Note that the style of the \bibitem labels (in []) is slightly
%% different from previous examples.  The natbib system solves a host
%% of citation expression problems, but it is necessary to clearly
%% delimit the year from the author name used in the citation.
%% See the natbib documentation for more details and options.


\end{document}

%% End of file `sample.tex'.
